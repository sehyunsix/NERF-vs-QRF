\quad 양자컴퓨팅은 중첩, 얽힘 등 양자역학의 물리적 현상을 활용하여 기존의 고전 컴퓨터보다 우월한 성능을 내기를 기대하는 정보처리 방식이다. 0 혹은 1의 bit 상태를 가지는 고전 컴퓨터와 달리, 양자 컴퓨터는 확률적으로 0과 1이 중첩된 Quantum Bit, \textbf{Qubit}을 사용하며, 이를 적절히 활용하여 다양한 방식으로의 효율적인 연산을 꾀하고 있다. 예컨대 Deutsch–Jozsa algorithm은 기존의 고전 컴퓨터로는 \( O(2^N) \)의 시간복잡도를 가지는 문제를 단 \( O(1) \)만에 풀어내었고, Shor's Algorithm의 경우 기존에 지수 시간이 걸리던 소인수분해 문제를 다항 시간만에 풀어내어 현재 가장 널리 쓰이는 RSA 암호화 체계에 큰 위협을 주기도 했다. 이처럼 양자컴퓨팅을 사용하여 기존의 컴퓨터가 해결하기 어려워하는 문제를 모두 해결할 수 있지 않겠냐는 기대가 있지만, 안타깝게도 이것이 사실이라고 말하기는 어렵다. 이 물음은 모든 NP 문제를 양자컴퓨터로 다항 시간 내에 풀어낼 수 있는지의 물음으로 귀결되고, 이는 P, NP 문제와 그 사이 어딘가에 있을 BPP, BQP의 알려지지 않은 관계를 명확히 제시하라고 하는 물음과 본질적으로 같다. 분명한 것은, 기존의 컴퓨터가 해결하기 어려워하는 문제를 양자컴퓨터로는 빠르게 할 수 있는 특정 문제가 존재한다는 것이다. 그러므로 어떤 문제에 대해서 이것이 성립하는지, 어떤 알고리즘을 통해서 가능케 하는지 등등에 대한 주제는 높은 연구가치를 지닌다.

 % \indent
양자컴퓨팅 연구 분야의 대표적인 갈래는 (보편적인 컴퓨터가 그러하듯) 하드웨어와 소프트웨어가 있다. 하드웨어의 경우 물리적으로 qubit의 기능을 어떻게 실재적으로 구현할 것인지의 논의로부터 출발하는 연구이고, 초전도체를 기반으로 사용하는 IBM, 구글 등의 기업과 이온 트랩 방식을 사용하는 IonQ 등의 기업이 양자컴퓨터 하드웨어를 활발히 연구 중에 있다. 양자컴퓨팅의 소프트웨어 연구 분야는 양자 프로그래밍 언어를 통해 프로그래밍한 양자 알고리즘을 다양한 방식의 양자 하드웨어에 이식 가능(portable)토록 하는 것이며, 대표적인 양자컴퓨팅 프레임워크로는 IBM사에서 개발한 Qiskit, Xanadu사에서 개발한 PennyLane 등이 있다. 특별히, 양자 알고리즘, 양자 통신과 같은 양자적인 정보처리가 수학적으로 어떻게 고전 정보처리보다 성능적으로 우위에 있는지를 분석하는 연구 분야를 \textbf{양자정보이론}이라고 한다.
% \\ \indent

\textit{현재 머신러닝의 역량이 일상생활에서까지 입증되고 있는 가운데, 고전 컴퓨터에서 해결하는 데에 연산 비용이 막대한 회귀분석과 서포트 벡터 머신(Support Vector Machine) 등의 문제를 지수함수적으로 빠르게 해결하기를 기대하는 양자머신러닝 또한 양자정보이론 분야 중 중요한 이슈이다. } 특별히, 최근 NISQ 시대에 활용 가능한 Quantum-Classical Hybrid Machine Learning Method인 \textit{Variational Quantum Eigensolver}/Algorithm을 통해 양자화학의 여러 문제를 해결하고자 하는 시도가 있었다. 이처럼 어떤 현실적인 문제를 양자머신러닝으로 해결할 수 있을 것인지, 이때 양자머신러닝에서 필수적인 양자데이터 인코딩 방식, 모델 구성 방식, 정보 전처리/후처리 방식 등의 세세한 주제가 모두 중요한 연구 주제라고 할 수 있다.

% \\ \indent
모든 양자알고리즘(양자머신러닝 포함)의 연산 과정은 크게 데이터 인코딩, 양자 회로, 그리고 measurement operation을 통한 정보 추출로 이루어진다. 이때 데이터 인코딩, 양자 회로에 사용되는 양자 게이트는 모두 선형대수의 Unitary Matrix로 이루어진다. 이는 당연히 Linear하기에, 머신러닝에서 Activation Function을 통해 Non-Linearity를 부여하는 것이 양자컴퓨팅의 연산에서는 자연스럽지 않다. 본 연구에서는 이러한 양자머신러닝의 한계점을 극복하는 Quantum-based Perceptron을 관찰하고, 이를 확장한 Quantum Multi-Perceptron의 방법론을 고안하여 이를 통해 Quantum MLP의 가능성을 제시한다.