% \section{Quantum Perceptron with Quantum Activation Function}
우리는 양자 머신러닝의 큰 진입장벽 중 하나가 비선형성의 구현이라는 것을 이전 장에서 발견했다.
양자 알고리즘의 각 step, 즉 각 게이트를 통과하는 프로세스 하나하나는 본질적으로 각 게이트에 대응되는 행렬을 벡터에 곱해 주는 작업이다.
행렬곱은 선형 연산이다. 즉 양자 알고리즘은 선형 연산이다. 결국 양자 알고리즘으로 완전한 비선형성의 구현이 불가능하다는 결론에 이르게 된다. 이것은 우리의 과제에 있어 큰 장벽이다.
인공지능, 그중에서도 MLP(Multi Layer Perceptron)는 행렬 연산의 반복으로 구성되는데, 행렬 연산의 차원을 높이기 위해 각 Layer 사이에 비선형 연산을 추가하는 것이 필수적이기 때문이다.
해당 문제의 해결이 선행되기 전까지 MLP를 양자 알고리즘만으로 완전히 구현하는 것은 불가능하다는 것을 어렵지 않게 떠올릴 수 있다.
그렇기에 고전 알고리즘과 양자 알고리즘을 적절히 섞어 사용하는 Hybrid Algorithm이라는 대안이 존재하나, 양자 알고리즘만으로 머신 러닝의 과정을 모두 구현하겠다는 시도는 분명 학술적으로 그 가치가 충분하다.

우리는 상기한 비선형성 구현 문제의 해결 방법 하나를 제시한 논문을 발견하였다.
해당 논문에는 단층 퍼셉트론의 연산 프로세스를 양자 회로로 구현하는 방법이 제시되어 있다.
단층 퍼셉트론은 크게 \(N_{in} \times 1\) 크기의 input vector \(\vec{x}\), 
\(\vec{x}\)의 각 성분과 곱해지게 될 가중치 값이 저장되어 있는 \(N_{in} \times 1\) 크기의 weight vector \(\vec{w}\), 
스칼라 값인 bias \(b\), 그리고 비선형성을 위한 활성화 함수(Activation Function, AF) \(f\)로 구성되어,
\(\vec{x}\)가 input으로 주어지면 \(f(\vec{w}\cdot\vec{x}+b)\)의 값을 내놓는다.
해당 논문은 \(\vec{w}\cdot\vec{x}+b\)의 계산뿐만 아니라, \(f(\vec{w}\cdot\vec{x}+b)\)의 계산 또한 양자 알고리즘으로 구현한다.
\(f\) 함수 자체는 사용할 수 없으나, 양자 회로를 잘 구성하여 \(f(\vec{w}\cdot\vec{x}+b)\)의 값을 근사적으로 구하는 방법론을 제시한 것이다.

우리는 해당 논문을 이번 연구의 중심 축으로 결정하였다.
우리는 해당 논문을 이해하고, 수학적 과정을 코드로 구현하여 직접 확인해 보았다.
또한 논문에서는 제대로 논의하고 있지 않은 부분인 MLP의 양자적 구현을 고민하던 중, 우리는 매우 효율적인 Multiperceptron의 양자 알고리즘을 구현하는 데에 성공하였고, 이를 코드로 구현하여 잘 작동함을 확인하였다.

이번 장은 해당 논문에 대한 우리의 심도 깊은 이해와 구현, 그리고 우리가 발전시킨 부분에 대해 서술되어 있다.
4.1장에서는 \(f(\vec{w}\cdot\vec{x}+b)\)를 양자 알고리즘으로 구현하는 과정을 따라가고, 그 과정을 직접 코드로 구현해 본 과정을 공유한다.
이후 4.2장에서는 우리의 독창적이고 효율적인 Multiperceptron의 구현 아이디어에 대해 설명한다.

\section{Imparting Nonlinearity for Quantum Perceptron}

\subsection{Summarization of the Notations}

이번 Section은 앞으로의 내용을 잘 이해하기 위해 기호들이나 용어의 의미를 미리 정리하고 가기 위한 장이다.

앞으로 qubit의 개수를 표기하는 영어 소문자에 대해, 대문자 표기는 그 qubit에 대응되는 힐베르트 공간의 차원을 나타낸다.
즉 qubit 개수가 n인 경우, N = \(2^n\)이다.

임의의 1개 qubit과 대응되는 2차원 힐베르트 공간 \(\mathcal{H}\)에 대하여, register q의 n개의 qubit과 대응되는 \(2^n\)차원 힐베르트 공간을 \(\mathcal{H}_q^{\otimes n} \equiv \mathcal{H}_{q_{n-1}} \otimes \mathcal{H}_{q_{n-2}} \otimes \dots \otimes \mathcal{H}_{q_{0}}\) 라고 한다.
이때 register는 qubit이 저장되어 있는 공간의 이름을 말한다. 간단하게 이해하면 qubit의 집합이라고 생각하면 된다.
\(\mathcal{H}\)의 computational basis가 \(\{ \ket{0}, \ket{1} \}\)임을 일전에 이야기했다. 그렇기에 \(\mathcal{H}_q^{\otimes n}\)의 computational basis는 \(\{ \ket{s_{n-1}s_{n-2}\dots s_0} : s_k \in \{0,1\}, k = 0,1,\dots , n-1\}\)의 이진수 형태로 나타내어지고, 각 원소는 이를 십진수 형태로 변환한 \(\{ \ket{i}, i \in \{0,1,\dots,2^n-1\} \} \)의 형태로 나타낼 수도 있다.
예컨대, \(\ket{N-1} \equiv \ket{2^n-1} \equiv \ket{111\dots 1} \equiv \ket{1}^{\otimes n}\)가 된다.

n개의 qubit으로 구성된 register q에 적용되는 연산 U를 표기하는 방법은, U가 separable인지 아닌지에 따라 나뉜다.
sperable이 아닌 경우에는 U가 q에 가해지는 연산임을 나타내기 위해 \(U_q\)라고 작성하거나, 아니면 단순히 U라고 적는다.
한편, U가 one-qubit transformation \(U_{q_j}\)들의 텐서곱으로 나타내어지는, 즉 separable인 경우에는 U를 \(U_q^{\otimes n} \equiv U_{q_{n-1}} \otimes U_{q_{n-2}} \otimes \dots \otimes U_{q_{0}}\)라고 적는다.

n개의 qubit으로 구성된 register q, d개의 qubit으로 구성된 register a를 생각하자. 두 레지스터는 N+D차원 Hibert Space로 합쳐질 수 있다. 이 합쳐진 Space를 \(\mathcal{H}_a^{\otimes d} \otimes \mathcal{H}_q^{\otimes n}\)로 작성한다. 이때 이 Space의 computational basis는 \(\{ \ket{i}_a \ket{j}_q : i = 0,\dots,D-1, j = 0,\dots,N-1\}\)이다.
간결함을 위해, 연산 O가 만약 둘 중 하나의 register에서만 작동하는 경우에 \(\mathbb{1}_a \otimes O_q, O_a \otimes \mathbb{1}_q\)를 각각 \(O_q, O_a\)로 간단하게 나타낸다. 이때 \(\mathbb{1}_a = I_{2^d}\)를 의미한다.
특히, \(\ket{i}\bra{i}_q \equiv \mathbb{1}_a \otimes \ket{i}_q {_q}\bra{i}\)는 register q의 \(\ket{i}\) state로의 D차원 projection을 나타낸다.

마지막으로, controlled gate(특정 qubit의 상태에 따라 연산의 작동 여부가 결정되는 게이트)를 나타내기 위해 사용하는 표기들을 정리해 보자. \(C_{q_i}U_{q_j}\)라는 표기는 \(q_i\) qubit의 state가 \(\ket{1}\)일 때 \(q_j\) qubit에 연산 U를 가하겠다는 의미이다.
이와 반대로 \(\bar{C_{q_i}}U_{q_j}\)은 \(q_i\) qubit의 state가 \(\ket{0}\)일 때 \(q_j\) qubit에 연산 U를 가하겠다는 의미이다.
즉 \(\bar{C_{q_i}}U_{q_j}\)은 \(X_{q_i}C_{q_i}U_{q_j}X_{q_i}\)와 동일한 연산이다. (X: Pauli-X gate)
만약 d개의 qubit으로 구성된 register a의 모든 qubit을 control qubit으로 사용하는 경우에는 \(C_a^dU_{q_j}\)라는 표기를 사용한다.

\subsection{Implementation of \(\vec{w}\cdot\vec{x}+b\)}

단일 퍼셉트론의 구현을 위해 n+d개의 qubit이 필요하다. n개의 qubit은 q register에, d개의 qubit은 a register에 들어 있다. input vector \(\vec{x}\)의 크기는 \(N_{in}\)으로 쓴다.
우선은 각 input값을 scaling하여 각 성분을 -1과 1 사이의 값으로 고정시키는 것으로 시작한다.
즉, \(\vec{w} \in [-1,1]^{N_{in}}, \vec{x} \in [-1,1]^{N_{in}}, b \in [-1,1]\)이 주어졌을 때, \(\vec{w}\cdot\vec{x}+b\)의 값을 계산하는 양자 회로의 구성 과정을 따라가 보겠다.
이 장과 그 다음 장은 본질적으로 우리의 논문 탐독 과정을 따라가기 위한 것이기에, 논문의 주요 Theorem을 직접 인용하고, 그 의미와 증명의 흐름을 간단하게 서술하는 것으로 구성하였다. 논문 원문에는 증명 전체가 서술되어 있으므로,
자세한 이해를 원한다면 Reference를 참고할 수 있다.

%\begin{lemma}
%    \(\vec{x}, \vec{w}, b\) 가 주어졌을 때, \(\displaystyle \bra{N-1}U_z(\vec{x},\vec{w},b)\ket{0} = \frac{\vec{x}\cdot \vec{w}+b}{N_{in}+1} \equiv z\) 를 만족하는 Unitary transformation \(U_z(\vec{x},\vec{w},b)\) 의 역할을 하는 양자 회로를 만들 수 있다.
%\end{lemma}

\begin{lemma}

    Given two vectors $\vec{x}, \vec{w} \in [-1, 1]^{N_{\text{in}}}$ and a number $b \in [-1, 1]$, and given a register of $n$ qubits such that $N = 2^n \geq N_{\text{in}} + 3$, then there exists a quantum circuit realizing a unitary transformation $U_z(\vec{x}, \vec{w}, b)$ such that
\[
\langle N - 1 | U_z(\vec{x}, \vec{w}, b) | 0 \rangle = \frac{\vec{w} \cdot \vec{x} + b}{N_{\text{in}} + 1} \equiv z
\]
where $\ket{0} \equiv \ket{0}^{\otimes n}$ and $\ket{N-1} \equiv \ket{1}^{\otimes n}$.

\end{lemma}

우선, \(\vec{w}\cdot \vec{x} + b\)의 구현을 위한 양자 회로의 구성 요소 중 필수적인, \(U_z(\vec{x},\vec{w},b)\) 부분의 기능을 구현하기 위한 Lemma가 제시된다.
대략적인 증명의 흐름은 다음과 같다.

\begin{pf}(sketch)

Lemma에서 \(\ket{0}, \ket{N-1}\)은 각각 \(\ket{0}^{\otimes n}, \ket{1}^{\otimes n}\)임을 기억하자.
\(\vec{v}_x = (\vec{x},1,A_x,\vec{0}), \vec{v}_{w,b} = (\vec{w},b,\vec{0},A_{w,b})\) 를 만들자.
이때 두 벡터는 \(n = \lceil\log_2{N_{in}+3}\rceil\)에 대해 \(N = 2^n\)의 크기를 가진다. 벡터의 성분으로 들어 있는 0의 개수는 N과 \(N_{in}\)값에 따라 결정된다.
또한, 두 벡터는 \(|\vec{v}_x| = |\vec{v}_{w,b}| = \sqrt{N_{in}+1}\)를 만족해야만 한다. \(A_x, A_{w,b}\)의 값은 해당 규칙에 맞게 결정해 주면 된다.
그러면 \(\vec{v}_{w,b}^T\vec{v}_{x} = \vec{w}\cdot\vec{x}+b \in [-N_{in}-1, N_{in}+1]\)이 자연스럽게 도출된다.

%두 개의 state \(\ket{\psi_{x}},\ket{\psi_{w,b}}\)를 각각 \(\ket{\psi_{x}} = \sum_{i=0}^{N-1}\frac{v_{x,i}}{\sqrt{N_{in}+1}}\ket{i}, \ket{\psi_{w,b}} = \sum_{i=0}^{N-1}\frac{v_{w,b,i}}{\sqrt{N_{in}+1}}\ket{i}\)라 정의하자.
%그러면 \(\bra{\psi_x}\ket{\psi_{w,b}} = \frac{\vec{w}\cdot\vec{x}+b}{N_{in}+1} \equiv z\)가 성립한다.

위에서 만든 두 벡터 \(\vec{v}_x, \vec{v}_{w,b}\)로 \(\bra{\psi_x}\ket{\psi_{w,b}} = \frac{\vec{w}\cdot\vec{x}+b}{N_{in}+1} \equiv z\)가 성립하는 두 개의 state \(\ket{\psi_{x}},\ket{\psi_{w,b}}\)를 만들 수 있다.

\(U_x = \mathcal{U}(\vec{v}_x), U_{w,b} = X^{\otimes n}, \mathcal{U}^\dag(\vec{v}_{w,b})\)라고 작성할 수 있다.
이때 \(\bra{\psi_{w,b}}\ket{\psi_x} = \bra{\psi_{w,b}}U_{w,b}^{\dag} U_{w,b}\ket{\psi_x} = \bra{N-1}U_{w,b}U_x\ket{0}^{\otimes n}\)이 성립하므로, Lemma에서 말하는 \(U_z(\vec{x},\vec{w},b)\)가 \(U_{w,b}U_x = X^{\otimes n}\mathcal{U}^\dag (\vec{v}_{w,b})\mathcal{U}(\vec{v}_x)\)임을 확인할 수 있고, 이것으로 증명이 완료된다.
\end{pf}

우리가 원하는 양자 회로의 주요 구성 요소 중 하나가 완성되었으므로, 이제 전체 기능을 담당하는 부분을 만들 차례이다. 그것이 가능하다는 것을 다음 Theorem이 말하고 있다.

\begin{theorem}
    
Let $z$ be the real value in the interval $[-1, 1]$ assumed by $\frac{\vec{w} \cdot \vec{x} + b}{N_{\text{in}} + 1}$, where $\vec{x}, \vec{w} \in [-1, 1]^{N_{\text{in}}}$ and $b \in [-1, 1]$. Let $q$ and $a$ be two registers of $n$ and $d$ qubits respectively, with $N = 2^n \geq N_{\text{in}} + 3$. Then there exists a quantum circuit which transforms the two registers from the initial state $\ket{0}_a \ket{0}_q$ to a $(n + d)$-qubit entangled state $\ket{\psi_z^d}$ of the form
\[
\ket{\psi_z^d} = \ket{\psi_z^d}_\perp + \frac{1}{2^{d/2}} \ket{z}_a^{\otimes d} \ket{N - 1}_q
\]
where
\[
\braket{N - 1 | \psi_z^d}_q = 0
\]
and
\[
\ket{z} \equiv \ket{0} + z \ket{1}.
\]

The circuit is expressed by $S_V X^{\otimes n}_q$ (Fig. 3a) where $X$ is the quantum NOT gate and
\[
S_V = V_{d-1} \cdots V_0
\]
with
\[
V_m = C_a^m U_z(\vec{x}, \vec{w}, b)_q C_a^m X_q^{\otimes n} C_q^n H_a^m, \quad m = 0, 1, \ldots, d - 1.
\]
\end{theorem}

이 theorem의 존재 가치는 두 register에 동시에 영향을 끼치는 어떠한 회로를 가지고 우리가 원하는 state \(\ket{\psi_z^d}\)를 얻어낼 수 있음을 보이는 것이다.
theorem에서 나타나는 \(\ket{\psi_z^m}_\perp\)의 경우
\(\ket{\psi_z^m} \in \mathcal{H}_{a_{m-1}} \otimes \mathcal{H}_{a_{m-2}} \otimes \dots \otimes \mathcal{H}_{a_{0}} \otimes \mathcal{H}_{a}^{\otimes n}\)에 대하여
\(\ket{N-1}\bra{N-1}_q\ket{\psi_z^m}_\perp = 0\)을 만족하도록 \(\ket{\psi_z^m}\)을 \(\ket{\psi_z^m}_\perp + \ket{\psi_z^m}_{||}\)로 분리하는 것으로 얻어진다.
이떄 \(\ket{N-1}\bra{N-1}_q\)은 projection 변환으로, Gram-Schmidt Process에 의해 \(\ket{\psi_z^m}_\perp\)의 존재성이 증명되어 있다.
증명의 대략적인 흐름은 다음과 같다.

\begin{pf}(sketch)

    \(V_m = C_{a_m}U_z(\vec{x},\vec{w},b)_qC_{a_m}X_q^{\otimes n}C_q^nH_{a_m}\)으로 잡을 수 있다. 여기서 \(a_m\)에 의해 작동 여부가 결정되는 \(U_z(\vec{x},\vec{w},b)_q\)는 위의 Lemma에서 결정된 그것이다.
    이제, \(\ket{\psi_z^m}_{||} = \frac{1}{2^{d/2}}\ket{z}_a^{\otimes d}\ket{N-1}_q\)라는 사실을 생각하고, \(V_m\)을 state \(\ket{0}_{a_m}\ket{\psi_z^m} = \ket{0}_{a_m}\ket{\psi_z^m}_\perp + \frac{1}{\sqrt(2)}(\ket{0}_{a_m}+\ket{1}_{a_m})\ket{\psi_z^m}_{||}\)에 적용하여
    \(\ket{\psi_z^{m+1}}\)을 얻게 된다. 이제, 이 과정을 각 qubit에 적용하는 것을 반복하는 연산 \(S_V = V_{d-1}\dots V_1V_0\)을 state \(\ket{0}_a^{\otimes d} \ket{N-1}_q\)에 적용하면, 결과물로 \(\ket{\psi_z^d}\)가 나온다.

\end{pf}

마지막으로 위의 Theorem에서 얻은 state \(\ket{\psi_z^d}\)를 measure하지 않고도 z의 값을 사용할 수 있음을 보여야 한다.
그것을 위해 아래의 Corollary가 제시된다. 그 내용과 간단한 증명의 흐름은 아래와 같다.

\begin{corollary}
    The state $\ket{\psi_z^d}$ stores as probability amplitudes all the powers $z^k$, for $k = 0, 1, \ldots, d$, up to a trivial factor. Indeed equation in Theorem above implies
\[
q \bra{N - 1}_a \bra{2^k - 1} \ket{\psi_z^d} = 2^{-d/2} z^k, \quad k = 0, 1, \ldots, d
\]
\end{corollary}

\begin{pf}(sketch)
    \(\ket{\psi_z^m}_\perp + \ket{\psi_z^m}_{||}\)라는 사실을 활용하면
    Theorem 1의 식 \(\ket{\psi_z^d}_q = \ket{\psi_z^d}_\perp + \frac{1}{2^{d/2}}\ket{z}_a^{\otimes d}\ket{N-1}_q\)에서 \(_q\bra{N-1}_a\bra{2^k-1}\ket{\psi_z^d} = 2^{-d/2}z^k\) 임을 유도할 수 있다.
\end{pf}

이상으로 첫 번째 단계에 대한 구현과 그 유효성에 대한 증명이 완료되었다.

\subsection{Implementation of Activation function}

이제 남은 것은 Activation function의 구현이고, 우리가 핵심적으로 관찰해야 했던 과정이다.
양자 알고리즘으로 \(f(\vec{w}\cdot\vec{x}+b)\)를 구현할 수 있음을 보이고 그 방법론의 타당성을 설명하기 위해서 논문은 크게 두 가지의 단계를 거친다.
우선은 양자 회로로 쉽게 구현할 수 있는 형태의 함수를 결정하고,
어떠한 형태의 함수 f에 대하여, z를 input으로 하면 f(z)의 값을 내놓는 양자 회로를 제시하는 것이다.
그리고 임의의 d차 테일러 전개가 존재하는 함수(analytic 함수)에 대하여, 그 함수의 d차 테일러 전개 \(f_d\)를 일전에 결정한 형태로 작성할 수 있음을 보인다.
a register의 qubit 수와 그 양자 회로가 내놓는 f의 taylor expansion의 차수가 동일하다는 사실에 주목하자.
이 과정을 통해 양자 회로로 임의의 analytic 함수의 d차 taylor expansion \(f_d\)에 대하여, \(f_d(z)\)의 값을 양자 회로를 통해 계산할 수 있음을 보이는 방식이다.
각 과정은 이하의 theroem과 corollary로 선언되고, 그 유효성이 증명되어 있다.

\begin{theorem}
    Let $\{f_k, k = 1, \ldots, d\}$ be the family of polynomials in $z$ defined by the following recursive law:
\[
f_k(z) = f_{k-1}(z) \cos \vartheta_{k-1} - z^k \sin \vartheta_{k-1}, \quad k = 1, \ldots, d,
\]
with $f_0(z) = 1$ and $\vartheta_k \in \left[-\frac{\pi}{2}, \frac{\pi}{2}\right]$ for any $k = 0, \ldots, d-1$.

Then there exists a family $\{U_k, k = 1, \ldots, d\}$ of unitary operators such that
\[
\bra{0}_a U_k \ket{z}_a^{\otimes d} = f_k(z).
\]

These unitary operators are, in turn, defined by the recursive law:
\[
U_k = C_{a_k} X_{a_k} \overline{C}_{a_k} R_y(-2 \vartheta_{k-1})_{a_0} U_{k-1}, \quad k = 1, \ldots, d,
\]
with $U_0 = \mathbb{1}$.

\end{theorem}

theorem에서 제시된 형태로 정의되어 있는 함수 \(f_k\)에 대하여 이전까지의 과정에 이어 \(f(z)\)값을 얻을 수 있음을 말하고 있다.
2단계의 회로는 \(U_d\)이다. 1단계 과정의 전체 회로를 \(S_VX_q^{\otimes n}\)로 나타내었기 때문에, 표기의 통일성을 위해 전체 회로 그림에서 \(U_d\) 대신 \(S_U\)라는 표기를 사용한다. 둘은 같은 것을 지칭한다.
간단한 증명의 흐름은 다음과 같다.

\begin{pf}(sketch)
    
    이부분 그냥 식조작이라 뭐라써야될질 모르겠음ㅠ 좀더 고민해봄
\end{pf}

\begin{corollary}
    
Let $f$ be a real analytic function over a compact interval $I$. If $f_d$ is the top member of the family of polynomials defined in \textbf{Theorem 2}, then the angles $\vartheta_k, k = 0, \ldots, d-1$ can be chosen in such a way that $f_d(z)$ coincides with the $d$-order Taylor expansion of $f(z)$ around $z = 0$, up to a constant factor $C_d$ which depends on $f$ and on the order $d$ as
\[
C_d = a_k \prod_{j=k}^{d-1} (\cos \vartheta_j)^{-1},
\]
where $a_k = \frac{1}{k!} f^{(k)}(0)$ is the first non-zero coefficient of the expansion and $f^{(k)}$ the $k$-order derivative of $f$.

\end{corollary}

이 corollary는 위의 theorem에서 말하는 형태로 임의의 analytic function의 d차 taylor expansion을 표현할 수 있음을 시사한다.
이때 analytic function은 테일러 급수가 f로 수렴하는 함수를, 즉 taylor expansion을 통해 \(f(z)\)값을 근사적으로 구할 수 있는 함수를 말한다.
예를 들어 sigmoid 함수는 analytic이므로, 제시된 방법론에서 Activation function으로 사용할 수 있을 것이다.
증명에서는 \(\theta_k\)의 값을 결정하는 방법이 서술되어 있다. 우리는 코드를 통해 그 방법이 모든 analytic function에서 유효함을 확인할 수 있었다.
Theorem과 Corollary를 종합하면, 우리는 Activation function이 analytic일 때, \(f(z)\)값을 taylor expansion을 통해 근사적으로 구할 수 있다는 사실을 알 수 있다.

여기까지가 논문에서 제시한 \(f(\vec{w}\cdot\vec{x}+b)\) 계산의 양자 알고리즘 구현 방법이다.
우리는 이 방법론에 대한 수학적 증명 과정을 상세히 이해하였고, 그것의 유효성을 각 theorem을 코드로 구현해 봄으로서 다시 한 번 확인하였다. 해당 코드는 이곳에서 확인해 봃 수 있다.

\section{Implementation of Multiperceptron with Superb Efficiency}
Some possible future works. (maybe 0.5\thru3 pages?)